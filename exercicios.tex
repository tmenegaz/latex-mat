\section{Exercícios}

Numa seção do TRE trabalhavam $ 32 $ funcionário dando atendimento ao público. A razão entre o número de homens e o número de mulheres, nessa ordem, é de 3 para 5. É correto afirmar que nessa seção o atendimento é dado por:

\begin{multicols}{2}
\begin{enumerate}[label=\alph*)]
	\item ( ) 20 homens e 12 mulheres;
	\item ( ) 18 homens e 14 mulheres;
	\item ( ) 16 homens e 16 mulheres;
	\item (X) 12 homens e 20 mulheres;
	\item ( ) 10 homens e 22 mulheres;
\end{enumerate}
\columnbreak

$ \dfrac{32}{m} = \dfrac{3+5}{5} = 8m = 160 = 20m$\\

$ \dfrac{32}{h} = \dfrac{3+5}{3} = 8h = 96 = 12h $
\end{multicols}

Em uma revenda de carros seminovos, a razão entre o número de carros pretos e o número de carros prata vendidos durante um mês foi de $ \dfrac{a}{b}$, com $ a $ e $ b $ sendo números naturais. Sabendo-se que no mês o número de carros vendidos foi $ v $, assinale o que for correto.

\begin{multicols}{2}
	\begin{enumerate}[label=\alph*)]
		\item (v) Se $ a = 3, b = 11 $ e $ v = 168 $, pode-se concluir que o número de carros prateados vendidos nesse mês superou o número de carros pretos em 96;
		\item (f) Se $ a = 5, b = 12 $ e o número de carros prateados vendidos nesse mês totaliza 60, então $ v = 204 $;
		\item (v) O número de carros pretos vendidos nesse mês é igual a $ \dfrac{av}{a+b} $;
		\item (v) Se $ a = 13, b = 5 $ e o número de carros pretos vendidos nesse mês totaliza 143, então $ v = 198 $;
	\end{enumerate}

	\columnbreak
	\begin{enumerate}[label=\alph*)]
	\item $ \dfrac{168}{c_ta} = \dfrac{3+11}{11} = 14c_ta = 1848 = 132$
	
	Carros pratas($ c_ta $): $132$.\\ $168-132 = 36$ carros pretos\\ $ 132 - 36 = 96c_ta$ à mais.
	
	\item $ \dfrac{204}{5+12} = 12k$
	
	$5 \cdot 12 = 60$ carros pretos;\\
	$12 \cdot 12 = 144$ carros pratas.
	
	\item $ \dfrac{3 \cdot 168}{3+11} = 36$ carros pretos
	
	\item  $ \dfrac{198}{5+13} = 11k$
	
	$13 \cdot 11 = 143$ carros pretos;\\
\end{enumerate}
\end{multicols}

Um homem dá um salto de $ 0,4m $ para cima, ao mesmo tempo que uma pulga dá um pulo de $ 400mm $. A razão entre os saltos é?

$ 0,4m \longrightarrow 40cm $ e $ 400mm \longrightarrow 40cm $. A razão é $ \dfrac{40}{40} = 1 $
