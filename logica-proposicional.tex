
\section{Lógica formal e a Dança}

Imagine o seguinte caso hipotético:
Em um ambiente, de sala de aula, para Dança contemporânea, foi proposto aos estudantes caminharem pela sala observando as pessoas e os espaços entre elas. O objetivo é desperta nos participantes a sensibilidade de perceber o outro no espaço: as dimensões de cada um; como o copo se inclina de um lado para o outro, na troca de peso de uma perna para a outra, ao caminhar; como cada um se sente por saber que está sendo observado e observando; e, manter o ritmo da caminhada; entre outros. Essa proposta pode trazer elementos suficientes para a construção de um raciocínio lógico formal? Sabemos que a lógica formal pode representar \textbf{afirmações} que fazemos em linguagem cotidiana para apresentar \textbf{fatos} ou \textbf{transmitir afirmações}. Sendo assim, é possível obter de um ambiente de formação em Dança \textbf{afirmações propositivas} que possam servir como instrução inicial para um AI? Supondo, então que sim, eu vou tentar demonstrar, com base no exemplo acima algumas \textbf{proposições} ou \textbf{declarações} que poderiam vir ser elaboradas.

Para esse exercício vou especificar as pessoas com nomes inventados. Então, podemos ter proposições com valor \textbf{verdadeiro} ou \textbf{falso} para a interpretação lógica do contexto da atividade de Dança(conhecimento específico disponível).

Então algumas proposições podem ser elaboradas supondo que as pessoas podem ser especificadas, pois pronomes como ele, ela, aquele aquela não podem ser usados para elaborar \textbf{sentenças lógicas}.
 
Com relação a observação de uma pessoa sob as \textbf{``dimensões de cada um''}

\begin{enumerate}[label=\alph*)]
	\item João é alto e magro.
	\item Luiza é magra ou bonita.
	\item Ana é gorda e bonita.
	\item Tiago é baixo ou feio.
\end{enumerate}

Com relação a observação de uma pessoa sob \textbf{``como o copo se inclina de um lado para o outro, na troca de peso de uma perna para a outra, ao caminhar''}

\begin{enumerate}[label=\alph*)]
	\item João tem os passos largos ou abre a perna direita.
	\item Luiza tem passos pequenos e abre a perna direita.
	\item Ana tem passos pequenos e fecha a perna esquerda.
	\item Tiago fecha a perna direita e tem passos médios.
\end{enumerate}

Com relação a observação de uma pessoa sob \textbf{``como cada um se sente por saber que está sendo observado e observando''}

\begin{enumerate}[label=\alph*)]
	\item João gosta de observar ou não gosta de ser observado.
	\item Luiza não gosta de observar ou prefere ser observada.
	\item Ana gosta de ser observada e de observar.
	\item Tiago não gosta de ser observado ou de observar.
\end{enumerate}

Com relação a observação de uma pessoa sob \textbf{``manter o ritmo da caminhada''}

\begin{enumerate}[label=\alph*)]
	\item João caminha rápido se for baixo.
	\item Luiza caminha de vagar se for grande.
	\item Ana caminha com passos largos se não for baixa.
	\item Tiago é baixo quando caminha de vagar.
\end{enumerate}

Agora, relacionando as áreas do conhecimento para a ação de um agente inteligente podemos inferir que ter ciência da lógica proposicional é fundamental para a condução dessa investigação, pois segundo o Dr. Silvio do Lago Pereira(https://www.ime.usp.br/~slago/ia-2.pdf): \\
IA estuda como simular \textbf{comportamento inteligente} $ \longrightarrow $ comportamento inteligente é o resultado de \textbf{raciocínio} correto sobre \textbf{conhecimento} específico disponível $ \longrightarrow $ conhecimento e raciocínio podem ser representados em \textbf{lógica} $ \longrightarrow $ o formalismo lógico  mais simples é a \textbf{lógica proposicional}.\\
Por meio desse processo formal será possível alimentar o AI, pois a elaboração do raciocínio sobre a Dança deverá passar pela representação do comportamento inteligente do ambiente para a lógica proposicional.
 
Voltando ao caso hipotético. É importante considerar que cada uma dessas afirmações de valor falso ou verdadeiro, a respeito das pessoas, tem o ponto de vista subjetivo do observador. O que indica $ \textbf{2}^\textbf{n} $ possibilidades para cada proposição.

Agora imagine quantos tipos de afirmações uma pessoa pode fazer, em uma roda de diálogo, que podem ser verificadas pelo professor de Dança. Com um olhar atento o professor pode obter declarações com algum valor lógico sobre a tarefa ou fazer perguntas para extrair afirmações. O objetivo dessas verificações poderiam ser obter mais informações que viessem a contribuir para a construção de atividades fundamentadas no desempenho da turma ou de cada um. Então para $ \textbf{n} $ preposições (sendo $ \textbf{n} $ a quantidade de preposições)  o que se tem é:

\begin{itemize}
	\item  $n = 2 \longrightarrow 2^2 = 4 $ possibilidades de verdadeiro ou falso; 
	\item  $n = 3 \longrightarrow 2^3 = 8 $ possibilidades de verdadeiro ou falso;
	\item  $n = 4 \longrightarrow 2^4 = 16 $ possibilidades de verdadeiro ou falso;
	\item $ \vdots $
\end{itemize}


Talvez em uma turma com um ou dois alunos a tarefa de análise e/ou registro de afirmações seja possível, mesmo que tediosa. Ou quem sabe ainda, nem todos os cruzamentos de dados para gerar informações fossem percebidos. Mas em um ambiente com muitos alunos a tarefa de, coletar, selecionar e medir dados por meio de afirmações declarativas, sem o auxílio de um AI, seria impossível.

Imagine ainda em uma aula de Dança com um ambiente diferente: balé clássico, dança de salão, sapateado, release etc. os objetivos e tudo o mais que se quer em cada aula, trariam fatos distintos. Então, para não perder todos esses dados e as possíveis informações obtidas do cruzamento deles é razoável contar com o auxílio de um AI.

Mas por que realizar uma análise lógica para nutrir um AI em atividades tão subjetivas que tem grande chance de mudar entre uma realização e outra? Talvez para tentar ter mais dados sobre a incerteza, ou ter evidência de que as pessoas não são tão diferentes na execução de uma determinada atividade, ou ainda negar as propostas anteriores, ou os dados obtidos das atividades de Dança sirvam para pesquisas em outras ciências, ou, quem sabe, realizar medições que possam levar a descoberta de novas perguntas. Com esses motivos, considero todas essas possibilidades como sendo importantes para saber como implementar um AI no processo de formação em Dança.

\pagebreak

\section{Conectivos}

\begin{multicols}{2}
	\setlength{\columnseprule}{.5pt}
A \textbf{conjunção} conecta duas ou mais preposições por meio do $ \textbf{e} $ que tem como símbolo o $ \wedge $.\\

\begin{tabular}{|c|c|c|}
	\hline 
	\multicolumn{3}{|c|}{conjunção $ \wedge $} \\ 
	\hline 
	A	& B & $ A \wedge B $ \\ 
	\hline 
	v	& v & v \\ 
	\hline 
	v	& f & f \\ 
	\hline 
	f	& v & f \\ 
	\hline 
	f	& f & f \\ 
	\hline 
\end{tabular} 
	
	\columnbreak
		
A \textbf{disjunção} conecta duas ou mais preposições por meio do $ \textbf{ou} $ que tem como símbolo o $ \vee $.\\

\begin{tabular}{|c|c|c|}
	\hline 
	\multicolumn{3}{|c|}{disjunção $ \vee $} \\ 
	\hline 
	A	& B & $ A \vee B $ \\ 
	\hline 
	v	& v & v \\ 
	\hline 
	v	& f & v \\ 
	\hline 
	f	& v & v \\ 
	\hline 
	f	& f & f \\ 
	\hline 
\end{tabular} 

\end{multicols}

\subsection{Exemplo}
Encontre o valor lógico para as expressões abaixo

	\begin{multicols}{2}
		\setlength{\columnseprule}{.5pt}
		\begin{enumerate}[label=\alph*)]
		\item $ A \wedge B \vee C $
		
		\begin{tabular}{|c|c|c|c|c|}
			\hline 
			\multicolumn{5}{|c|}{ $ A \wedge B = P \therefore P \vee C $} \\ 
			\hline 
			A	& B & C & $ A \wedge B $ & $ P \vee C $ \\ 
			\hline 
			v	& v & v & v & v \\ 
			\hline 
			v	& v & f & v & v \\ 
			\hline 
			v	& f & v & f & v \\ 
			\hline 
			v	& f & f & f & f \\ 
			\hline 
			f	& v & v & f & v \\
			\hline 
			f	& v & f & f & f \\
			\hline 
			f	& f & v & f & v \\
			\hline 
			f	& f & f & f & f \\
			\hline 
		\end{tabular} \\
		
		\columnbreak
		\item $ A \vee B \wedge A $ 
		
		\begin{tabular}{|c|c|c|c|}
			\hline 
			\multicolumn{4}{|c|}{ $ B \wedge A = Q \therefore A \vee Q $} \\ 
			\hline 
			A	& B & $ B \wedge A $ & $ A \vee Q $ \\ 
			\hline 
			v	& v & v & v \\ 
			\hline 
			v	& f & f & v \\ 
			\hline 
			f	& v & f & f \\
			\hline 
			f	& f & f & f \\
			\hline 
		\end{tabular} \\
\end{enumerate}
	\end{multicols}

\pagebreak

\begin{multicols}{2}
	\setlength{\columnseprule}{.5pt}
A \textbf{condicional} implica que a verdade de uma preposição(A) leva a verdade da outra preposição(B) que tem como símbolo o $ \rightarrow $.\\

\begin{tabular}{|c|c|c|}
	\hline 
	\multicolumn{3}{|c|}{condicional $ \rightarrow $} \\ 
	\hline 
	A	& B & $ A \rightarrow B $ \\ 
	\hline 
	v	& v & v \\ 
	\hline 
	v	& f & f \\ 
	\hline 
	f	& v & v \\ 
	\hline 
	f	& f & v \\ 
	\hline 
\end{tabular} \\

\textbf{A} é a preposição \textbf{antecedente} e \textbf{b} é \textbf{consequente}. Quando \textbf{A} for \textbf{v} e \textbf{B} for \textbf{f} temos uma contradição, pois \textbf{A} uma condição suficiente para \textbf{B}. 

\columnbreak

A \textbf{bicondicional} é uma abreviação de $ A \rightarrow B \wedge B \rightarrow A $  que tem como símbolo o $ \leftrightarrow $.\\

\begin{tabular}{|c|c|c|}
	\hline 
	\multicolumn{3}{|c|}{bicondicional $ \leftrightarrow $} \\ 
	\hline 
	A	& B & $ A \leftrightarrow B $ \\ 
	\hline 
	v	& v & v \\ 
	\hline 
	v	& f & f \\ 
	\hline 
	f	& v & f \\ 
	\hline 
	f	& f & v \\ 
	\hline 
\end{tabular} \\

\textbf{A} é condição suficiente é suficiente para \textbf{B}.
\end{multicols}

\subsection{Exemplo}
Encontre o valor lógico para as expressões abaixo

\begin{multicols}{2}
	\setlength{\columnseprule}{.5pt}
	\begin{enumerate}[label=\alph*)]
		\item $ A \wedge B \rightarrow C $
		
		\begin{tabular}{|c|c|c|c|c|}
			\hline 
			\multicolumn{5}{|c|}{ $ A \wedge B = P \therefore P \rightarrow C $} \\ 
			\hline 
			A	& B & C & $ A \wedge B $ & $ P \rightarrow C $ \\ 
			\hline 
			v	& v & v & v & v \\ 
			\hline 
			v	& v & f & v & f \\ 
			\hline 
			v	& f & v & f & v \\ 
			\hline 
			v	& f & f & f & v \\ 
			\hline 
			f	& v & v & f & v \\
			\hline 
			f	& v & f & f & v \\
			\hline 
			f	& f & v & f & v \\
			\hline 
			f	& f & f & f & v \\
			\hline 
		\end{tabular} \\
		\columnbreak
		
		\item $ A \leftrightarrow B \wedge A $ 
		
		\begin{tabular}{|c|c|c|c|}
			\hline 
			\multicolumn{4}{|c|}{ $ B \wedge A = Q \therefore A \leftrightarrow Q $} \\ 
			\hline 
			A	& B & $ B \wedge A $ & $ A \leftrightarrow Q $ \\ 
			\hline 
			v	& v & v & v \\ 
			\hline 
			v	& f & f & f \\ 
			\hline 
			f	& v & f & v \\
			\hline 
			f	& f & f & v \\
			\hline 
		\end{tabular} \\
	\end{enumerate}
\end{multicols}

\pagebreak

Esses conectivos são conhecidos como \textbf{binários} pois juntam duas expressões por meio de um conectivo lógico que pode ser $ \wedge, \vee, \rightarrow $ e $ \leftrightarrow $.

Essas expressões podem ser aninhadas para tentar resolver problemas extraídos do cotidiano e para atender a formalidade da lógica é necessário seguir uma \textbf{ordem de precedência}.

\begin{enumerate}
	\item a expressão dentro do parenteses mais interno;
	\item a negação: $ \neg $ ;
	\item conjunção: $ \wedge $;
	\item disjunção: $ \vee $;
	\item condicional: $ \rightarrow; $
	\item bicondicional: $ \leftrightarrow $.
\end{enumerate}

Entretanto, muitas vezes para resolver um problemas é necessário usar parênteses para determinar qual é o \textbf{conectivo principal}. Um conetivo principal sempre será o último conectivo de uma expressão na ordem de precedência.

Na expressão \\

$ A \wedge (B \rightarrow C)\neg $, \\

o conectivo principal é $ \wedge $ e o primeiro no ordem de precedência é $ \rightarrow $. \\

Já em \\

$ ((A \vee B) \wedge C) \rightarrow (B \vee C\neg) $ \\

o conectivo principal é $ \rightarrow $ e o primeiro na ordem de precedência é $ \neg $. \\

É possível atribuir um nome para cada expressão das extremidades do conectivo principal: \\

$ ((A \vee B) \wedge C) = P $ e \\

$ (B \vee C\neg) = Q $ \\

Isso quer dizer que $ P \rightarrow Q $ é o mesmo que $ ((A \vee B) \wedge C) \rightarrow (B \vee C\neg) $

\section{Tautologia, Contradição e Equivalência}

Em uma \textbf{tautologia} o valor da expressão será sempre verdadeiro como em $ A \vee A\neg $

\noindent
Em uma \textbf{contradição} o valor de uma expressão será sempre falso como em $ A \wedge A\neg $

\noindent
Quando são utilizadas duas expressões que resultam em valores lógicos iguais dizemos que as expressões são \textbf{equivalentes} na forma de $ P \Leftrightarrow Q $

\noindent
Com isso podemos ter uma equivalência de tautologias onde $ P = (A \rightarrow B) $ \\ e $ Q = (A\neg \rightarrow B\neg) $:

A equivalência de $ P \leftrightarrow Q $ é uma tautologia pois $ (A \rightarrow B) \Leftrightarrow (A\neg \rightarrow B\neg) $.

Para $ P = (A \wedge A\neg) $  e $ Q = 0 $ então $ P \leftrightarrow Q $ é uma tautologia pois $ (A \wedge A\neg) \Leftrightarrow 0 $ é uma equivalência entre duas contradições. \\

\section{Lógica proposicional}

Usamos a lógica formal para criar representações simbólicas de afirmações do cotidiano e agora faremos uso dela para chegar a conclusões a partir de premissas. Dito de outro modo, o que se quer é $ Q $ como uma conclusão lógica de $ P_i \cdots P_n $ sempre que a verdade das preposições $ P_i \cdots P_n $ implicar na verdade de $ Q $: \\

$ P_1 \wedge P_2 \wedge P_3 \wedge \cdots \wedge P_n \rightarrow Q $\\

É importante ficar atendo no fato de que a condicional deverá ser verdadeira com base na relação entre a conclusão e a hipótese, e não em qualquer conhecimento incidental que venha a existir sobre $ Q $. Ademais um argumento só será válido quando for uma \textbf{tautologia}. Para isso é possível usar a tabela verdade, algoritmo de tautologia ou regra de dedução.





