\section{Potenciação}
Definição de potenciação:\\

$ a $ e $ m \in \mathbb{R} $ de tal forma que $ a^m = \mathbb{R}$.\\

Exemplos:

\begin{enumerate}[label=\alph*)]
\item $ a^{1} = a $
\item $ a^{0} = 1 $
\item $ 0^{0} = $ \textquestiondown (indeterminação) mas no assunto cálculo o $ 0^{0} = 1$
\item $ a^{m} = \underbrace{a \cdot a \cdot a \cdot a \ldots a}_{n\  fatores} $
\item $ 3 + 3 + 3 = 9 $
\item $ 3 \cdot 3 = 9 $
\item $ 3^{4} = 3\cdot 3\cdot 3\cdot 3 = 81 $
\item $ 3^{2} \cdot 3^{2} \Longrightarrow 3^{2+2} = 3^{4} $
\item $ 2^{2} + 2^{3} \Longrightarrow 1 \cdot 2^{2} + 2 \cdot 2^{2} \Longrightarrow 3 \cdot 2^{2} = 12 $
\item $ 4^{3} - 4^{2} \Longrightarrow 4^{1+2} - 4^{2} \Longrightarrow (4 \cdot 4^{2}) - 4^{2} \Longrightarrow 4^{2}(4 - 1) = 3 \cdot 4^{2} = 48 $
\item $ (-2)^{3} = (-2)\cdot (-2)\cdot (-2) = -8 $ (com expoente ímpar o resultado é negativo) 
\item $ (-2)^{4} = (-2)\cdot (-2)\cdot (-2)\cdot (-2) = 16 $
\item $ -2^{4} = -(2\cdot 2\cdot 2\cdot 2) = -16 $
\item $ 2^{3} + 3^{2} = 8 + 9 = 17 $
\item $ 2^{3} - 3^{2} = 8 - 9 = -1 $
\item $ 2^{3} + (-3)^{2} = 8 + 9 = 17 $
\item $ (-2)^{3} + 3^{2} = -8 + 9 = 1 $
\item $ -2^{3} - (-3)^{2} = -8 - 9 = -17 $
\item $ 2^{3} \cdot 3^{2} = 8 \cdot 9 = 72 $
\item $ \dfrac{3^{2}}{2^{3}} = 3^{2}\cdot \dfrac{1}{2^{3}} \longrightarrow 9 \cdot \dfrac{1}{8} = \dfrac{9}{8}$
\end{enumerate}

\subsection{Propriedades da potenciação}
\begin{enumerate}[label=P\roman*)]
	\item $ a^{m} \cdot a^{n} = a^{m+n} \longrightarrow 2^{3} \cdot 2^{2} = 2^{3+2} = 2^{5} = 32 $
	\item $ (a^{m})^{n} = a^{m\cdot n} \longrightarrow (2^{3})^{2} = 2^{3}\cdot 2^{3} = 2^{3+3} $ ou $ 2^{3\cdot 2} = 2^{6} = 64 $
	\item $ a^{-m} = \dfrac{1}{a^{m}}\ (OBS: a \neq 0)$
	\begin{enumerate}[label=\alph*)]
		\item $ 3^{-2} = \dfrac{1}{3^{2}} = \dfrac{1}{9}; $
		
		\item $\dfrac{1}{2^{-3}} = 2^{3} = 8 $ 
	\end{enumerate}
	\item $ \dfrac{a^{m}}{a^{n}} = a^{m-n} \longrightarrow \dfrac{3^{5}}{3^{2}} = 3^{5}\cdot \dfrac{1}{3^{2}} \longrightarrow 3^{5}\cdot 3^{-2} = 3^{5+(-2)} = 3^{3} = 27$
	\item $ a^{m}\cdot b^{m}= (a\cdot b)^{m} \longrightarrow 2^{3}\cdot 3^{3} = (2\cdot 3)^{3} = 6^{3} = 216 $
	\item $ \dfrac{a^{m}}{b^{m}} = (\dfrac{b}{a})^{m} \longrightarrow (\dfrac{8^{3}}{2^{3}}) = (\dfrac{8}{2})^{3} = 4^{3} = 64 $
	\item $ (\dfrac{a}{b})^{-m} = (\dfrac{b}{a})^{m} $ (OBS: deriva das $ Piii $ e $ Piv $)
	\begin{enumerate}[label=\alph*)]
		\item $ (\dfrac{3}{2})^{-3} = \dfrac{3^{-3}}{2^{-3}} \Longrightarrow \dfrac{1}{3^{3}}\cdot \dfrac{1}{2^{-3}} \Longrightarrow \dfrac{1}{3^{3}}\cdot \dfrac{2^{3}}{1} \Longrightarrow \dfrac{2^{3}}{3^{3}} = (\dfrac{2}{3})^{3} $
		
		\item $ (\dfrac{2}{4})^{-3} = \dfrac{2^{-3}}{4^{-3}} \Longrightarrow \dfrac{4^{3}}{2^{3}} = (\dfrac{4}{2})^{3}  = 2^{3} = 8 $
		
		\item $ (5 \div 3)^{-2} = (\dfrac{5}{1} \cdot \dfrac{1}{3})^{-2} \Longrightarrow (\dfrac{1}{5} \cdot \dfrac{3}{1})^{2} = (\dfrac{3}{5})^{2} $
		
		\item $ (12 \div 6)^{-2} = (\dfrac{2}{1})^{-2} \Longrightarrow (\dfrac{1}{2})^{2} $
	\end{enumerate}
\end{enumerate}

\subsection{Questões}
\begin{enumerate}[label=\alph*)]
	\item Sendo $ x $ e $ y $ números reais não nulos, a expressão $ (x^{-2} + y^{-2})^{-1} $ é o mesmo que:

	\indent \indent \textbf{Resposta}:
	
	\indent \indent $ (\dfrac{1}{x^{2}} + \dfrac{1}{y^{2}})^{-1} \Longrightarrow (\dfrac{y^{2} + x^{2}}{x^{2}\cdot y^{2}})^{-1} \Longrightarrow \dfrac{x^{2} \cdot y^{2}}{y^{2} + x^{2}} = \dfrac{(x \cdot y)^{2}}{y^{2} + x^{2}} $
	
	\item A fração $ \dfrac{2^{98} + 4^{50} - 8^{34}}{2^{99} - 32^{20} + 2^{101}} $ é igual a qual fração?
	
	\indent \indent \textbf{Resposta}:
	
	\indent \indent $ \dfrac{2^{98}}{2^{99}} = 2^{98 - 99} = 2^{-1} = \dfrac{1}{2} $ \\

	\indent \indent $ \dfrac{4^{50}}{-32^{20}} \Longrightarrow \dfrac{4^{50}}{-(8\cdot 4)^{20}} \Longrightarrow \dfrac{(2^{2})^{50}}{-(2^{3}\cdot 2^{2})^{20}} \Longrightarrow \dfrac{2^{2\cdot 50}}{-(2^{3 + 2})^{20}} \Longrightarrow \dfrac{2^{100}}{-(2^{5\cdot 20})} = \dfrac{2^{100}}{-2^{100}} $ \\
	
	\indent \indent $ \dfrac{-8^{34}}{2^{101}} \Longrightarrow \dfrac{-(2^{3})^{34}}{2^{101}} \Longrightarrow \dfrac{-2^{3\cdot 34}}{2^{101}} \Longrightarrow \dfrac{-2^{102}}{2^{101}} $ \\
	
	\indent \indent $ \dfrac{2^{98} + 2^{100} - 2^{102}}{2^{99} - 2^{100} + 2^{101}} \Longrightarrow \dfrac{2^{98}(1 + 2^{2} - 2^{4})}{2^{99}(1 - 2^{1} + 2^{2})} \Longrightarrow \dfrac{1(1 + 4 - 16)}{2(1 - 2 + 4)} \Longrightarrow \dfrac{1 \cdot -11}{2\cdot 3} = \dfrac{-11}{6}  $
	
	\item O valor de $ 0,2^{3} \cdot 1,6^{2} $ é:
	
	\indent \indent \textbf{Resposta}:

	\indent \indent $ \dfrac{2}{10}^{3} \cdot \dfrac{16}{10}^{2} \Longrightarrow \dfrac{2}{10}^{3} \cdot ((\dfrac{2}{10}^{2})^{2})^{2} \Longrightarrow \dfrac{2}{10}^{3+8} = \dfrac{2}{10}^{11} $ 
	
	\item Encontre a solução para a expressão numérica $ \dfrac{4^{2} + (5 - 3)^{2}}{(9 - 7)^{2}} $
	
	\indent \indent \textbf{Resposta}:
	
	\indent \indent $ \dfrac{(2^{2})^{2} + 2 ^{2}}{2^{2}} \Longrightarrow \dfrac{\cancel{2^{2}}(2^{2} + 1)}{\cancel{2^{2}}} \Longrightarrow 4 + 1 = 5 $

\end{enumerate}
	
	\textbf{Reduza a uma potência}
\begin{enumerate}[label=\alph*)]
	
	\item $ (-2^{2})^{2} $
	
	\indent \indent \textbf{Resposta}: $ (-2^{2})(-2^{2}) \Longrightarrow 2^{2+2} = 2^{4} $\\
	
	\item $ \dfrac{4}{8} $
	
	\indent \indent \textbf{Resposta}: $ \dfrac{2^{2}}{2^{3}} \Longrightarrow 2^{2-3} = 2^{-1} $\\
	
	\item $ 5^{2}\cdot 5^{5}\cdot 5^{-1} $
	
	\indent \indent \textbf{Resposta}: $ 5^{2+5} \cdot \dfrac{1}{5} \Longrightarrow 5^{7} \cdot \dfrac{1}{5} \Longrightarrow \dfrac{5^{7}}{5} \Longrightarrow 5^{7-1} = 5^{6} $ \\	
	
	\item $ \dfrac{3^{6}\cdot 3^{-2}}{3^{4}} $
	
	\indent \indent \textbf{Resposta}: $ \dfrac{3^{6-2}}{3^{4}} \Longrightarrow 3^{6 - 2}\cdot 3^{-4} \Longrightarrow 3^{6-2-4} = 3^{0} = 1 $ \\
	
	\item $ \sqrt[4]{\dfrac{2^{65}+2^{67}}{10}} $
	
	\indent \indent \textbf{Resposta}: $ \sqrt[4]{\dfrac{2^{65}+2^{65}\cdot 2^{2}}{10}} \Longrightarrow \sqrt[4]{\dfrac{(1+4)2^{65}}{10}} \Longrightarrow \sqrt[4]{\dfrac{2^{65}\cdot 5}{2\cdot 5}} \Longrightarrow \sqrt[4]{\dfrac{2^{65}\cdot \cancel{5}}{2\cdot \cancel{5}}} \Longrightarrow \sqrt[4]{2^{65-1}} \Longrightarrow \sqrt[4]{2^{64}} = 2^{16} $
	

\end{enumerate}