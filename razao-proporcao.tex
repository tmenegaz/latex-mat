\section{Razões}
Considere um carro de corrida com $4m$ de comprimento e um Kart com $2m$ de comprimento. Para compararmos as medidas dos carros, basta dividir o comprimento de um deles pelo do outro. Assim: $\dfrac{4}{2} = 2$ Então, o tamanho do carro de corrida é $2$ vezes o tamanho do Kart. Podemos aformar também que o Kart tem a metade do tamanho do carro de corrida: $\dfrac{1}{2}$ \\
Agora podemos concluir que \textbf{a comparação entre dois números racionais, por meio de uma divisão, chama-se razão.} \\

A razão é dada de $3$ formas: \textbf{fracionária, percentual e decimal}. A razão também pode ser expressa com sinal negativo, desde que seus termos tenham sinais contrários \\

\subsection{Termos da razão}
Uma razão tem o quociente de $\dfrac{a}{b}$, $b \neq 0$ onde $a$ é o \textbf{antecedente} e $b$ é o \textbf{consequente}. Usa-se a razão para dividir, separar duas grandezas e se lê como "$a$ esta para $b$ ou $a$ para $b$".\\

$\dfrac{3}{5}$ A leitura da razão é $3$ está para $5$ ou $3$ para $5$. \\

\subsection{Razões inversas}
Considere as razões $\dfrac{2}{3}$ e $\dfrac{3}{2}$. O produtos dessas duas razões é $1$, nesse caso podemos afirmar que $\dfrac{2}{3}$ e $\dfrac{3}{2}$ são \textbf{razões inversas}: $\dfrac{2}{3} \cdot \dfrac{3}{2} = 1$.\\

Saiba que \textbf{duas razões são inversas entre si quando o produto delas é igual a $1$.} \\

\subsection{Exemplos}
\begin{enumerate}
	\item $200$g de café está para $2000$g de açúcar. Qual a razão do café para o açúcar em percentual para cada grama?\\
	
	$\dfrac{200g}{2000g}  = \dfrac{200 \div 200}{2000 \div 200} = \dfrac{1g}{10g}$ \\
	
	$
	\begin{array}{l l r r}
	\multicolumn{2}{r}{10} \vline & \multicolumn{2}{l}{10} \\ \cline{3-4}
	\multicolumn{2}{l}{-10} & \multicolumn{2}{l}{0,1} \\ \cline{1-2}
	\multicolumn{2}{r}{00} &  \\
	\end{array}
	$
	
	$0,1 \cdot 100 = 10\%$g de café em razão do açúcar
	
	\item Dos $1200$ inscritos num concurso, passaram $240$ candidatos. Qual a razão dos candidatos aprovados neste concurso?\\
	
	$\dfrac{240}{1200}  = \dfrac{240 \div 240}{1200 \div 240} = \dfrac{1}{5}$\\
	
	 $1$ aprovado para cada $5$ inscrito.\\
	
	\begin{multicols}{4}
		\indent MDC de $1200$ e $240 = 240$\\
		$\begin{array}{cc|cc}
		1200, & 240 & 2 & \\ 
		600, & 120 & 2 & \\ 
		300, & 60 & 2 & \\
		150, & 30 & 2 & \\
		75, & 15 & 3 & \\
		25, & 5 & 5 & \\
		5, & 1 & 5 & \\
		1, & 1 &  & 2^4 \cdot 3 \cdot 5 \Rightarrow MDC = 240 \\
		\hline
		\end{array}$
		\large
		\columnbreak
		$
		\begin{array}{r l}
		\multicolumn{2}{c}{_{1}} \vspace*{-5.5pt} \\ 
		\multicolumn{2}{c}{16} \\ 
		\multicolumn{2}{c}{\times3} \\ \cline{1-2}
		\multicolumn{2}{c}{48}
		\end{array}
		$
		\large
		\columnbreak
		$
		\begin{array}{r l}
		\multicolumn{2}{r}{_{4}} \vspace*{-5.5pt} \\ 
		\multicolumn{2}{r}{48} \\ 
		\multicolumn{2}{r}{\times5} \\ \cline{1-2}
		\multicolumn{2}{r}{240} \\
		\end{array}
		$
		\columnbreak
		$
		\begin{array}{r l r r}
		\multicolumn{2}{r}{240} \vline & \multicolumn{2}{l}{240} \\ \cline{3-4}
		\multicolumn{2}{l}{-240} & \multicolumn{2}{l}{1} \\ \cline{1-2}
		\multicolumn{2}{r}{000} &  \\
		\end{array}
		$
		\columnbreak
		$
		\begin{array}{l l r r}
		\multicolumn{2}{r}{1200} \vline & \multicolumn{2}{l}{240} \\ \cline{3-4}
		\multicolumn{2}{l}{-1200} & \multicolumn{2}{l}{5} \\ \cline{1-2}
		\multicolumn{2}{r}{0000} &  \\
		\end{array}
		$
	\end{multicols}

	\item Calcular a razão entre a altura de duas crianças, sabendo que a primeira possui uma altura de $h_1 = 1,20m$ e a segunda possui uma altura de $h_2 = 1,50m$. Qual a razão entre $h_1$ e $h_2$? \\
	
	$\dfrac{h_1}{h_2} = \dfrac{1,20\cancel{m}}{1,50\cancel{m}} = \dfrac{1,2 \cdot 10}{1,5  \cdot 10} = \dfrac{12 \div 3}{15 \div 3} = \dfrac{4}{5}$\\
	
	\item Um cubo de ferro de $1cm$ de aresta tem massa de $7,8g$. Determina a razão entre a massa e o volume desse corpo. O que se significa essa razão?
	
	volume: $1cm \cdot 1cm \cdot 1cm = 1cm^3$\\
	
	razão: $ \dfrac{7,8g}{1cm^3} = 7,8g/cm^3$ ou seja, $ 7,8 $ por centímetro cúbico.\\
	
	Essa razão se significa que cada $ 1cm^3 $ de ferro pesa $ 7,8g $.\\
	
	\item Simplifique as seguintes razões
	
	\begin{itemize}		
		
		\item $\dfrac{105}{120} = \dfrac{105 \div 15}{120 \div 15} = \dfrac{7}{8}$\\	
		\begin{multicols}{3}
		$\begin{array}{cc|cc}
		105, & 120 & 2 & \\ 
		105, & 60 & 2 & \\ 
		105, & 30 & 2 & \\
		105, & 15 & 3 & \\
		35, & 5 & 5 & \\
		7, & 1 & 5 & \\
		1, & 1 & 7 & \\
		1, & 1 &  & 3 \cdot 5 \Rightarrow MDC = 15 \\
		\hline 
		\end{array}
		$
		\columnbreak
		$
		\begin{array}{r l r r}
		\multicolumn{2}{r}{105} \vline & \multicolumn{2}{l}{15} \\ \cline{3-4}
		\multicolumn{2}{l}{-105} & \multicolumn{2}{l}{7} \\ \cline{1-2}
		\multicolumn{2}{r}{000} &  \\
		\end{array}
		$
		\columnbreak
		$
		\begin{array}{l l r r}
		\multicolumn{2}{r}{120} \vline & \multicolumn{2}{l}{15} \\ \cline{3-4}
		\multicolumn{2}{l}{-120} & \multicolumn{2}{l}{8} \\ \cline{1-2}
		\multicolumn{2}{r}{000} &  \\
		\end{array}
		$
	\end{multicols}

		\item $\dfrac{40}{12} = \dfrac{40 \div 4}{12 \div 4} = \dfrac{10}{3}$\\
		
		$\begin{array}{cc|cc}
		40, & 12 & 2 & \\ 
		20, & 6 & 2 & \\ 
		10, & 3 & 2 & \\
		5, & 3 & 3 & \\
		5, & 1 & 5 & \\
		1, & 1 &  & 2 \cdot 2 \Rightarrow MDC = 4 \\
		\hline
		\end{array}$\\
		
		\item $1,5$ e $ 3,2$\\
		
		$1,5 \Rightarrow \dfrac{1,5 \cdot 10}{1 \cdot 10} \Rightarrow \cfrac{15}{10} = \dfrac{15 \div 5}{10 \div 5} = \dfrac{3}{2}$\\
		
		
		$3,2 \Rightarrow \dfrac{3,2 \cdot 10}{1 \cdot 10} \Rightarrow \dfrac{32}{10} = \dfrac{32 \div 2}{10 \div 2} = \dfrac{16}{5}$ \\
		
			$\begin{array}{cc|cc}
		 32 & 10 & 2 & \\ 
		 16 & 5 & 2 & \\ 
		 8 & 5 & 2 & \\
		 4 & 5 & 2 & \\
		 2 & 5 & 2 & \\
		 1 & 5 & 5 & \\
		 1 & 1 &  & 2 \Rightarrow MDC = 2 \\
		 \hline  
		\end{array}$\\
		
		\item $\dfrac{15\%}{6\%} = \dfrac{15\% \div 3}{6\% \div 3} = \dfrac{5\%}{2\%}$
	\end{itemize}

\section{Proporção}

Proporção é uma igualdade entre duas razões.\\

Rogério e Cláudio passeiam com seus cachorros. Rogério pesa $120kg$ e seu cão $ 40kg $. Cládio, por sua vez, pesa $ 48kg $ e seu cão $ 16kg $. Qual a razão entre o peso dos $ 2 $ rapazes? Qual a razão entre o peso dos cachorros?\\

Rapazes: $ \dfrac{120kg}{48kg} = \dfrac{120kg \div 24}{48kg \div 24} = \dfrac{5}{2} $\\

\pagebreak

\begin{multicols}{3}
	$\begin{array}{cc|cc}
120 & 48 & 2 & \\ 
60 & 24 & 2 & \\ 
30 & 12 & 2 & \\
15 & 6 & 2 & \\
15 & 3 & 3 & \\
5 & 1 & 5 & \\
1 & 1 &  & 2^3 \cdot 3 \Rightarrow MDC = 24 \\
\hline 
\end{array}$\\
\columnbreak
$
\begin{array}{l l r r}
\multicolumn{2}{r}{120} \vline & \multicolumn{2}{l}{24} \\ \cline{3-4}
\multicolumn{2}{l}{-120} & \multicolumn{2}{l}{5} \\ \cline{1-2}
\multicolumn{2}{r}{000} &  \\
\end{array}
$

\columnbreak
$
\begin{array}{l l r r}
\multicolumn{2}{r}{48} \vline & \multicolumn{2}{l}{24} \\ \cline{3-4}
\multicolumn{2}{l}{-48} & \multicolumn{2}{l}{2} \\ \cline{1-2}
\multicolumn{2}{r}{00} &  \\
\end{array}
$
\end{multicols}

Cachorros: $ \dfrac{40kg}{16kg} = \dfrac{40kg \div 24}{48kg \div 24} = \dfrac{5}{2} $\\

\begin{multicols}{3}
	$\begin{array}{cc|cc}
	40 & 16 & 2 & \\ 
	20 & 8 & 2 & \\ 
	10 & 4 & 2 & \\
	5 & 2 & 2 & \\
	5 & 1 & 5 & \\
	1 & 1 &  & 2^3 \Rightarrow MDC = 8 \\
	\hline 
	\end{array}$\\
	\columnbreak
	$
	\begin{array}{l l r r}
	\multicolumn{2}{r}{40} \vline & \multicolumn{2}{l}{8} \\ \cline{3-4}
	\multicolumn{2}{l}{-40} & \multicolumn{2}{l}{5} \\ \cline{1-2}
	\multicolumn{2}{r}{00} &  \\
	\end{array}
	$
	
	\columnbreak
	$
	\begin{array}{l l r r}
	\multicolumn{2}{r}{16} \vline & \multicolumn{2}{l}{8} \\ \cline{3-4}
	\multicolumn{2}{l}{-16} & \multicolumn{2}{l}{2} \\ \cline{1-2}
	\multicolumn{2}{r}{00} &  \\
	\end{array}
	$
\end{multicols}

Verifica-se, então que as duas razões são iguais. Nesse caso é possível afirmar que a igualdade $ \dfrac{120}{48} = \dfrac{40}{16} $ é uma proporção. Outra forma de saber se duas razões são proporcionais é por meio da propriedade fundamental da proporção: \textbf{o produto dos meio é igual ao produto dos extremos}\\

$ \color{blue}a, \color{red}b, \color{red}c, \color{blue}d \color{black}= \dfrac{\color{blue}a}{\color{red}b} = \dfrac{\color{red}c}{\color{blue}d} = \color{blue}a \cdot \color{blue}d \color{black}= \color{red}b \cdot \color{red}c$ onde $ \color{blue}a $ e $ \color{blue}d $ são os \textbf{extremos} e $ \color{red}b $ e $ \color{red}c $ são os \textbf{meios}. Então, dada a proporção $ 3 $ está para $ 4 $ assim como $ 27 $ está para $ 36 $ os meios são: $ 4 $ e $ 27 $ e os extremos são $ 3 $ e $ 36 $\\

$ \dfrac{\color{blue}3}{\color{red}4} = \dfrac{\color{red}27}{\color{blue}36} = 3 \cdot 36 = 4 \cdot 27$;\\

$ 3 \cdot 36 = 108$;\\
$ 4 \cdot 27 = 108 $.\\

Numa salina, de cada metro cúbico ($ m^3 $) de água salgada, são retirados $ 40 dm^3 $ de sal. Para obtermos $ 3m^3 $ de sal, quantos metros cúbicos de água salgada são necessários?\\

$ dm^3 \Rightarrow m^3 = 40 \div 100 = 0,04m^3 \therefore \dfrac{1m^3}{0,04m^3}$ ($ 1m^3 $ para $ 0,04m^3 $ de sal)\\

$ \dfrac{1m^3}{0,04m^3} = \dfrac{x}{3m^3} = x0,04m^3 = 3m^3 $\\

$ x = \dfrac{3m^3}{0,04m^3}$\\

\begin{multicols}{3}
$ 3 \cdot 100 = 300$\\

\columnbreak
$ 0,04 \cdot 100 = 4 $\\
\columnbreak
$
\begin{array}{l l r r}
\multicolumn{2}{r}{300} \vline & \multicolumn{2}{l}{4} \\ \cline{3-4}
\multicolumn{2}{l}{-300} & \multicolumn{2}{l}{75} \\ \cline{1-2}
\multicolumn{2}{r}{000} &  \\
\end{array}
$
\end{multicols}

$ x = 75m^3 $

\subsection{Quarta proporcional}

dados $ 3 $ números racionais $ a $, $ b $ e $ c $, não nulos, determina-se quarta proporcional desse número o número $ x $\\
Determine o valor de $ x $ na proporção $ \dfrac{2}{7} = \dfrac{12}{x} $\\

$ 2x = 7 \cdot 12 = x =  \dfrac{84}{2} = x = 42 $

\subsection{Proporção contínua}

Quando os meio são iguais a proporção é chamada de contínua e o quarto termo é chamado de $ 3ª $ proporcional e, quando os termos são positivos, os termos iguais são ditos média proporcional ou geometria entre os extremos. Assim a proporção contínua de termos positivos é dada por: $ \dfrac{a}{b} = \dfrac{b}{c} $ onde $ b $ é a média proporcional e $ c $ é a $ 3ª $ proporcional.\\

Calcule a $ 3ª $ proporcional de $ 8 $ e $ 12 $, sendo $ 12 $ a média proporcional\\

\begin{multicols}{3}
	$ \dfrac{8}{12} = \dfrac{12}{x}$\\
	
	$8x = 12 \cdot 12$\\	
	
	$x = \dfrac{144}{8}$\\
	
	$x = 18$\\
	
	\columnbreak
	$
	\begin{array}{r r r}
	& \multicolumn{2}{r}{12} \\ 
	\times & \multicolumn{2}{r}{12} \\ \cline{1-3}
	& \multicolumn{2}{r}{24} \\
	\multicolumn{3}{c}{+12} \\ \cline{1-3}
	\multicolumn{3}{r}{144}
	\end{array}
	$
	\columnbreak

	$
	\begin{array}{r r r r}
	\multicolumn{2}{r}{144} \vline & \multicolumn{2}{l}{8} \\ \cline{3-4}
	\multicolumn{2}{l}{-8} & \multicolumn{2}{l}{18} \\ \cline{1-2}
	\multicolumn{2}{r}{64} &  \\
	\multicolumn{2}{l}{-64} &  \\\cline{1-2}
	\multicolumn{2}{r}{00} &  \\
	\end{array}
	$
\end{multicols}

Calcule a média proporcional entre $ 2 $ e $ 18 $.

\begin{multicols}{2}
	$ \dfrac{2}{x} = \dfrac{x}{18}$\\
	
	$x^2 = 2 \cdot 18$\\	
	
	$x = \sqrt{36} \Rightarrow x = \sqrt[\color{red}\cancel{\color{black}2}]{6\cancel{^2}}$\\
	
	$x = \abs{6}$\\
	
	\columnbreak
	$
	\begin{array}{r r}
	\multicolumn{2}{l}{_{+1}} \vspace*{-5.5pt} \\
	\multicolumn{2}{r}{18} \\ 
	\multicolumn{2}{r}{\times2} \\ \cline{1-2}
	\multicolumn{2}{r}{36} \\
	\end{array}
	$
\end{multicols}

\subsection{1ª propriedade}

Em uma proporção, a soma dos dois primeiros está para o $ 2º $ termo, assim como a soma dos dois últimos está para o $ 4º $:


$ \dfrac{a}{b} = \dfrac{c}{d} $

Então, adicionando $ 1 $ a cada membro da primeira proporção temos

$ \dfrac{a}{b} = \dfrac{c}{d} \Rightarrow \dfrac{a}{b} + 1 \Rightarrow \dfrac{a+b}{b}; \; \dfrac{c}{d} + 1 \Rightarrow \dfrac{c+d}{d}$\\

$ \dfrac{a}{b} + \dfrac{b}{b} = \dfrac{c}{d} + \dfrac{d}{d}$

Também é possível dizer que em uma proporção, a soma dos dois primeiros está para o $ 1º $ termo, assim como a soma dos dois últimos está para o $ 3º $:

$ \dfrac{b}{a} = \dfrac{d}{c} $

E ainda, adicionando $ 1 $ a cada membro da segunda proporção temos

$ \dfrac{b}{a} = \dfrac{d}{c} \Rightarrow \dfrac{b}{a} + 1 \Rightarrow \dfrac{b+a}{a}; \; \dfrac{d}{c} + 1 \Rightarrow \dfrac{d+c}{c}$\\

$ \dfrac{b}{a} + \dfrac{a}{a} = \dfrac{d}{c} + \dfrac{c}{c}$

Determine o valor de $ x $ e $ y $ na proporção $ \dfrac{x}{y} = \dfrac{3}{4} $, sabendo que $ x+y = 84 $.

\begin{multicols}{5}
	$ \dfrac{x}{y} = \dfrac{3}{4}$

	\columnbreak
	$\dfrac{84}{y} = \dfrac{x+y}{y};$
		
	$ \dfrac{3+4}{4} = \dfrac{c+d}{d}$
	
	\columnbreak
	$ \dfrac{84}{y} = \dfrac{3+4}{4} $
	
	$ 7y = 84 \cdot 4 $
	
	$ y = \dfrac{366}{7} $\\
	\columnbreak
	$
	\begin{array}{r r}
	\multicolumn{2}{l}{_{+1}} \vspace*{-5.5pt} \\
	\multicolumn{2}{r}{84} \\ 
	\multicolumn{2}{r}{\times4} \\ \cline{1-2}
	\multicolumn{2}{r}{336} \\
	\end{array}
	$
	\columnbreak
		$
	\begin{array}{r r r r}
	\multicolumn{2}{r}{336} \vline & \multicolumn{2}{l}{7} \\ \cline{3-4}
	\multicolumn{2}{l}{-336} & \multicolumn{2}{l}{48} \\ \cline{1-2}
	\multicolumn{2}{r}{000} &  \\
	\end{array}
	$
\end{multicols}
$ y = 48 $

\subsection{2ª propriedade}
Em uma proporção, a diferença dos dois primeiros está para o $ 2º $ termo, assim como a diferença dos dois últimos está para o $ 4º $:


$ \dfrac{a}{b} = \dfrac{c}{d} $

Então, adicionando $ -1 $ a cada membro da primeira proporção temos

$ \dfrac{a}{b} = \dfrac{c}{d} \Rightarrow \dfrac{a}{b} - 1 \Rightarrow \dfrac{a-b}{b}; \; \dfrac{c}{d} - 1 \Rightarrow \dfrac{c-d}{d}$\\

$ \dfrac{a}{b} - \dfrac{b}{b} = \dfrac{c}{d} - \dfrac{d}{d}$

Também é possível dizer que em uma proporção, a diferença dos dois primeiros está para o $ 1º $ termo, assim como a diferença dos dois últimos está para o $ 3º $:

$ \dfrac{b}{a} = \dfrac{d}{c} $

E ainda, adicionando $ -1 $ a cada membro da segunda proporção temos

$ \dfrac{b}{a} = \dfrac{d}{c} \Rightarrow \dfrac{b}{a} - 1 \Rightarrow \dfrac{b-a}{a}; \; \dfrac{d}{c} - 1 \Rightarrow \dfrac{d-c}{c}$\\

$ \dfrac{b}{a} - \dfrac{a}{a} = \dfrac{d}{c} - \dfrac{c}{c}$

Determine o valor de $ x $ e $ y $ na proporção $ \dfrac{x}{y} = \dfrac{5}{2} $, sabendo que $ x-y = 18 $.

\begin{multicols}{5}
	$ \dfrac{x}{y} = \dfrac{5}{2}$
	
	\columnbreak
	$\dfrac{18}{x} = \dfrac{x-y}{y};$
	
	$ \dfrac{5-2}{2} = \dfrac{c-d}{d}$
	
	\columnbreak
	$ \dfrac{18}{y} = \dfrac{5-2}{2} $
	
	$ 3y = 18 \cdot 2 $
	
	$ y = \dfrac{36}{3} $\\
	\columnbreak
	$
	\begin{array}{r r}
	\multicolumn{2}{l}{_{+1}} \vspace*{-5.5pt} \\
	\multicolumn{2}{r}{18} \\ 
	\multicolumn{2}{r}{\times2} \\ \cline{1-2}
	\multicolumn{2}{r}{36} \\
	\end{array}
	$
	\columnbreak
	$
	\begin{array}{r r r r}
	\multicolumn{2}{r}{36} \vline & \multicolumn{2}{l}{3} \\ \cline{3-4}
	\multicolumn{2}{l}{-36} & \multicolumn{2}{l}{12} \\ \cline{1-2}
	\multicolumn{2}{r}{00} &  \\
	\end{array}
	$
\end{multicols}
$ y = 12 $

\subsection{3ª propriedade}

Em uma proporção, a soma dos antecedentes está para a soma dos consequentes, assim como cada antecedente está para os eu consequente.

Considere a proporção $ \dfrac{a}{b} = \dfrac{c}{d} $, permutando os meios temos $ \dfrac{a}{c} = \dfrac{b}{d} $, aplica-se a $ 1ª $ propriedade $ \dfrac{a+c}{c} = \dfrac{b+d}{d} $, então, por fim, permuta-se os meios: $ \dfrac{a+c}{b+d} = \dfrac{c}{d} = \dfrac{a}{b} $\\

Sabe-se que $ a+b = 12 $, determine $ a $ e $ b $ na proporção $ \dfrac{a}{2} = \dfrac{b}{4} $

$ \dfrac{a+b}{2+4} = \dfrac{a}{2} = \dfrac{b}{4} $

\begin{multicols}{2}
$ \dfrac{12}{6} = \dfrac{a}{2} = 6a = 2 \cdot 12$\\

$ a = \dfrac{24}{6}$\\

$ a = 4 $
\columnbreak

$ \dfrac{12}{6} = \dfrac{b}{4} = 6b = 4 \cdot 12$\\

$ a = \dfrac{48}{6}$\\

$ a = 8 $
\end{multicols}


\subsection{4ª propriedade}

Em uma proporção, a diferença dos antecedentes está para a diferença dos consequentes, assim como cada antecedente está para os eu consequente.

Considere a proporção $ \dfrac{a}{b} = \dfrac{c}{d} $, permutando os meios temos $ \dfrac{a}{c} = \dfrac{b}{d} $, aplica-se a $ 2ª $ propriedade $ \dfrac{a-c}{c} = \dfrac{b-d}{d} $, então, por fim, permuta-se os meios: $ \dfrac{a-c}{b-d} = \dfrac{c}{d} = \dfrac{a}{b} $\\

Sabe-se que $ a-b = -24 $, determine $ a $ e $ b $ na proporção $ \dfrac{a}{5} = \dfrac{b}{7} $

$ \dfrac{a-b}{5-7} = \dfrac{a}{5} = \dfrac{b}{7} $

\begin{multicols}{2}
$ \dfrac{-24}{-2} = \dfrac{a}{5} = a = 5 \cdot \cancelto{12}{\dfrac{ -24}{-2}}$\\

$ a = 5 \cdot 12$\\

$ a = 60 $
\columnbreak

$ \dfrac{-24}{-2} = \dfrac{b}{7} = b = 7 \cdot \cancelto{12}{\dfrac{ -24}{-2}}$\\

$ b = 7 \cdot 12$\\

$ b = 84 $
\end{multicols}

\subsection{5ª propriedade}

Em uma proporção, o produto dos antecedentes está para o produto dos consequentes, assim como o quadrado da cada antecedente está para o quadrado do seu consequente.

Considere a proporção $ \dfrac{a}{b} = \dfrac{c}{d} $, multiplique os dois membros por $ \dfrac{a}{b} $, temos:\\ $ \dfrac{a}{b} \cdot \dfrac{a}{b} = \dfrac{c}{d} \cdot \dfrac{a}{b} \Rightarrow \dfrac{a^2}{b^2} = \dfrac{a \cdot c}{b \cdot d} $, então, por fim: $ \dfrac{a \cdot c}{b \cdot d} = \dfrac{a^2}{b^2} = \dfrac{c^2}{d^2} $\\

O Produto entre dois números é 56. Quais são esse números sabendo que a razão entre eles é de $ \dfrac{7}{8} $\\

$ a \cdot b = 56 $ e $ \dfrac{a}{b} = \dfrac{7}{8} $, \textbf{permutando} $ b $ com $ c $ tem-se $ \dfrac{a}{7} = \dfrac{b}{8} $ daí, pela regra vem:

$ \dfrac{a}{7} \cdot \dfrac{a}{7} = \dfrac{b}{8} \cdot \dfrac{a}{7} \Rightarrow \dfrac{a^2}{7^2} = \dfrac{b \cdot a}{8 \cdot 7} $, então, por fim: $ \dfrac{b \cdot a}{8 \cdot 7} = \dfrac{a^2}{7^2} $ ou $\dfrac{b \cdot a}{8 \cdot 7} = \dfrac{b^2}{8^2} $\\

Resolvendo $ a $

$ \dfrac{b \cdot a}{8 \cdot 7} = \dfrac{a^2}{7^2} \Rightarrow \dfrac{56}{56} = \dfrac{a^2}{49} \Rightarrow 56a^2 = 56 \cdot 49 \Rightarrow a^2 = \dfrac{\cancel{56} \cdot 49}{\cancel{56}} \Rightarrow a = \sqrt{49} \Rightarrow a = 7 $\\

Resolvendo $ b $

$ \dfrac{b \cdot a}{8 \cdot 7} = \dfrac{b^2}{8^2} \Rightarrow \dfrac{56}{56} = \dfrac{b^2}{64} \Rightarrow 56b^2 = 56 \cdot 64 \Rightarrow b^2 = \dfrac{\cancel{56} \cdot 64}{\cancel{56}} \Rightarrow x = \sqrt{64} \Rightarrow b = 8 $

\subsection{Proporção múltipla}

A proporção múltipla é uma série de razões iguais: $ \dfrac{2}{5} = \dfrac{4}{10} = \dfrac{6}{15} $

Dada uma série de razões iguais $ \dfrac{a}{b} = \dfrac{c}{d} = \dfrac{e}{f} $, de acordo com a $ 3ª $ e a $ 4ª $ propriedade, podemos escrever:

$ \dfrac{a+c+e}{b+d+f} = \dfrac{a}{b} = \dfrac{c}{d} = \dfrac{e}{f} $\\

$ \dfrac{a+c-e}{b+d-f} = \dfrac{a}{b} = \dfrac{c}{d} = \dfrac{e}{f} $\\

$ \dfrac{a-c+e}{b-d+f} = \dfrac{a}{b} = \dfrac{c}{d} = \dfrac{e}{f} $\\

$ \dfrac{a-c-e}{b-d-f} = \dfrac{a}{b} = \dfrac{c}{d} = \dfrac{e}{f} $\\

Na igualdade de razões $ \dfrac{x}{5} = \dfrac{y}{7} = \dfrac{z}{3} $, sabe-se que $ 2x - y + 3z = 36 $. O valor da soma de $ x + y + z $ é?\\

Conforme a propriedade temos que $ 2 \cdot 5 - 7 + 3 \cdot 3 = 12 $ daí $ \dfrac{36}{12} = 3 $\\

portanto: $ x = 5 \cdot 3 = 15 $; $ y = 7 \cdot 3 = 21 $; $ z = 3 \cdot 3 = 9 $. A soma de $ x + y + z $ é $ 45 $

\end{enumerate}


