	
 \section{Introdução a fração}
 	Sendo \textbf{a} e \textbf{b} números pertencentes a \textbf{$\aleph$} com \textbf{b} diferente de \textbf{0}, tem-se que $\dfrac{\textbf{a}}{\textbf{b}}$  (significa \textbf{a} dividido por \textbf{b}, ou simplesmente \textbf{a} $\div$ \textbf{b}), será um número \textbf{$\aleph$} se e somente se \textbf{a} for múltiplo de \textbf{b}. Em sendo diferente, a relação de \textbf{a} dividido por \textbf{b}, o número poderá ser \textbf{$\Re$}
 	
 	Então chamamos de $\dfrac{\textbf{a}}{\textbf{b}}$ de fração onde \textbf{a} é o numerador e \textbf{b} é o denominador. Nem sempre $\dfrac{\textbf{a}}{\textbf{b}}$ será um número natural pois a \textbf{ideia da fração é dividir em partes iguais} e dentre estas partes, consideraremos \textbf{umas} ou \textbf{algumas}, conforme o nosso interesse.
 	
 	As frações podem ser classificadas da seguinte forma, a saber:\\
 	
 	\indent \indent Fração \textbf{própria}: numerador $<$ denominador: $\dfrac{\textbf{2}}{\textbf{3}}$, $\dfrac{\textbf{1}}{\textbf{4}}$, $\dfrac{\textbf{3}}{\textbf{5}}$ \\
 	
 	\indent \indent Fração \textbf{imprópria}: numerador $\ge$ denominador: $\dfrac{\textbf{4}}{\textbf{3}}$, $\dfrac{\textbf{6}}{\textbf{5}}$, $\dfrac{\textbf{8}}{\textbf{5}}$ \\
 	
 	\indent \indent Fração \textbf{mista}: numerador $\geq$ denominador. Considerando-se duas frações: $\dfrac{\textbf{3}}{\textbf{3}}$ + $\dfrac{\textbf{4}}{\textbf{3}}$ é o mesmo que $ 1 + \dfrac{4}{3} $, onde o $ 1 $ é a simplificação de $\dfrac{\textbf{3}}{\textbf{3}}$. Tendo em vista outra fração mista, a foma de escreve pode ser apresentada assim: $\textbf{3}\dfrac{\textbf{4}}{\textbf{2}}$, ou, de forma detalhada:\\
 	
 	$\dfrac{\textbf{2}}{\textbf{2}} + \dfrac{\textbf{2}}{\textbf{2}} + \dfrac{\textbf{2}}{\textbf{2}} + \dfrac{\textbf{4}}{\textbf{2}} = \dfrac{\textbf{10}}{\textbf{2}}$. \\
 	
 	Observe que existe uma soma de frações com bases iguais e que, portanto, basta multiplicar o inteiro pelo denominador e somar ao numerador, preservando a base. \\
 	Ou seja, para $ 2\dfrac{3}{5} $ têm-se $ 5 \cdot 2 + 3 = 13 \longrightarrow \dfrac{13}{5} $. 
	Isso quer dizer que a fração $\dfrac{\textbf{5}}{\textbf{5}}$ será somada a $\dfrac{\textbf{5}}{\textbf{5}}$ e, em seguida, adicionada a fração $\dfrac{\textbf{3}}{\textbf{5}}$.
 	
 	O processo inverso é feito pela divisão da fração imprópria. Para transformar $ \dfrac{23}{6} $ em fração mista, faz-se assim: \\
 	
 	\begin{multicols}{2}[de fração imprópria $ \longrightarrow $ para fração mista]
	\setlength{\columnseprule}{.5pt}
 	$
 	\begin{array}{l l r r}
 	\multicolumn{2}{r}{23} \vline & \multicolumn{2}{c}{6} \\ \cline{3-4}
 	\multicolumn{2}{l}{-18} & \multicolumn{2}{l}{\color{red}3} \\ \cline{1-2}
 	\multicolumn{2}{r}{\color{blue}5} &  \\
 	\end{array}
 	$ 	
	\hrule
 	\columnbreak
	$
	\color{red}3 \dfrac{\color{blue}5}{\color{black}6} \color{black} \longrightarrow \color{red}3 \cdot \color{black}6 + \color{blue}5 = \color{black}23 \longrightarrow \dfrac{23}{6}
	$
	\newline
	\newline
	\hrule 
 \end{multicols}


\indent Fração \textbf{aparente}: numerador é múltiplo do denominador: $\dfrac{\textbf{4}}{\textbf{2}}$, $\dfrac{\textbf{6}}{\textbf{2}}$, $\dfrac{\textbf{24}}{\textbf{12}}$ \\
	 	
 	As \textbf{Frações Equivalentes} são frações que representam a mesma parte do todo. Veja que as frações $\dfrac{\textbf{1}}{\textbf{2}}$, $\dfrac{\textbf{2}}{\textbf{4}}$ e $\dfrac{\textbf{4}}{\textbf{8}}$, são equivalentes, pois multiplicando o numerador e o denominador por um número \textbf{$\aleph \neq 0$} tem-se o seu equivalente. Tomemos como exemplo a fração $\dfrac{\textbf{1}}{\textbf{2}}$ que tem o seu numerador e seu denominador multiplicados por $\textbf{2, 3, 4}$ e $\textbf{5}$: \\
 	
 	$\dfrac{1\cdot\textbf{2}}{2\cdot\textbf{2}} = \dfrac{2}{4} \longrightarrow \dfrac{1\cdot\textbf{3}}{2\cdot\textbf{3}} = \dfrac{3}{6} \longrightarrow \dfrac{1\cdot\textbf{4}}{2\cdot\textbf{4}} = \dfrac{4}{8} \longrightarrow \dfrac{1\cdot\textbf{5}}{2\cdot\textbf{5}} = \dfrac{5}{10}$ \\
 	
 	Conhecer as frações equivalente é importante para executar o processo de \textbf{simplificação de frações} até a obtenção de seu termo \textbf{irredutível}. Vajamos a fração $\dfrac{9}{12}$. O fator comum para $9$ e $12$ é o $\textbf{3}$. Então $\dfrac{9\div\textbf{3}}{12\div\textbf{3}} = \dfrac{3}{4}$ que é a fração irredutível (não possui fator comum). \\
 	
 	A partir do entendimento de frações equivalente podemos trabalhar com o número fracionário: $\dfrac{\textbf{m}}{\textbf{n}}$ com $\textbf{n} \neq \textbf{0}$ e $\dfrac{\textbf{m}}{\textbf{n}} \neq \aleph $. \\
 	
 	\noindent Então, para $\textbf{5}\cdot \textbf{x} = 1$ devemos pensar qual o valor de \textbf{x} para que o resultado da multiplicação por \textbf{5} seja \textbf{1}? Fica claro que \textbf{x} não pode ser um numero $\aleph $. Por isso devemos usar um \textbf{número fracionário}, qual seja $\dfrac{1}{5}$, pois $\textbf{5}\cdot \dfrac{\textbf{1}}{\textbf{5}} = \dfrac{\textbf{5}}{\textbf{5}} = 1$.
 	
 
 \subsection{Exemplo}
 \begin{enumerate}
 	
 	\item Na fração $\dfrac{\textbf{8}}{\textbf{2}}$ obtemos o quociente 4, então $\dfrac{\textbf{a}}{\textbf{b}}$ é um número natural e \textbf{8} é múltiplo de \textbf{2}. É uma fração \textbf{aparente}
 	
 	\item Raquel comeu $\dfrac{\textbf{3}}{\textbf{4}}$ de uma barra de chocolate. Isso significa que se dividirmos a barra em \textbf{4} partes iguais Raquel teria comido \textbf{3} partes e apenas uma teria sobrado.  É uma fração \textbf{própria}
 
 	\item Considere o retângulo
 		
 		\begin{tabular}{|c|c|c|c|}
 			\hline
 			\cellcolor{blue} & \cellcolor{blue} & & \cellcolor{blue} \\
 			\hline
 			& \cellcolor{blue} & \cellcolor{blue} & \\
			\hline
 		\end{tabular}
 	
 	\begin{enumerate}[label=\alph*)]
	 	\item Em quantas partes iguais o retângulo foi dividido?  dividido em $\textbf{8}$ partes.
	 	\item Cada uma destas partes representa que fração do retângulo?  $\space$ representa $\space \dfrac{\textbf{1}}{\textbf{8}}$
	 	\item A parte pintada representa que fração do retângulo? $\space$ representa $\space \dfrac{\textbf{5}}{\textbf{8}}$
	\end{enumerate}
		
 		\item Observe os retângulos abaixo e diga quanto representa cada parte da figura e a parte pintada
 		
 		\begin{enumerate}[label=\alph*)]
 		\item \begin{tabular}{|c|c|c|c|}
 			\hline
 			\cellcolor{yellow} & \cellcolor{yellow} & \cellcolor{yellow} & \cellcolor{yellow} \\
 			\hline
 			& \cellcolor{yellow} & \cellcolor{yellow} & \\
 			\hline
 			\cellcolor{yellow} & \cellcolor{yellow} & &\cellcolor{yellow} \\
 			\hline
 		\end{tabular}	
 	representa $\space \dfrac{\textbf{1}}{\textbf{12}}$ e a parte pintada $\space$ representa $\space \dfrac{\textbf{9}}{\textbf{12}}$
	 	
	 	\item \begin{tabular}{|c|c|c|}
	 		\hline
	 		\cellcolor{gray} & \cellcolor{gray} & \cellcolor{gray} \\
	 		\hline
	 		& \cellcolor{gray} & \cellcolor{gray} \\
	 		\hline
	 	\end{tabular}
	 	representa $\space \dfrac{\textbf{1}}{\textbf{6}}$ e a parte pintada $\space$ representa $\space \dfrac{\textbf{5}}{\textbf{6}}$
		
	\item \begin{tabular}{|c|c|}
		\hline
		\cellcolor{green} & \cellcolor{green} \\
		\hline
		\cellcolor{green} & \cellcolor{green} \\
		\hline
	\end{tabular}
representa $\space \dfrac{\textbf{1}}{\textbf{4}}$ e a parte pintada $\space$ representa $\space \dfrac{\textbf{4}}{\textbf{4}}$
	\end{enumerate}
	
	\item Para $\dfrac{\textbf{1}}{\textbf{6}}$ de uma pizza o custo é R\$$3,00$, quanto custa: \\
	
		\begin{enumerate}[label=\alph*)]
			\item $\dfrac{\textbf{3}}{\textbf{6}}$ da pizza? Custa R\$$9,00$ pois $\dfrac{\textbf{1}}{\textbf{6}} + \dfrac{\textbf{1}}{\textbf{6}} + \dfrac{\textbf{1}}{\textbf{6}} = \dfrac{\textbf{3}}{\textbf{6}}$ \\ \newline $\therefore 3,00 + 3,00 + 3,00 = 9,00$ ou ainda $3^{2}$ \\
			
			\item $\dfrac{\textbf{5}}{\textbf{6}}$ da pizza? Custa R\$$15,00$ pois $\dfrac{\textbf{1}}{\textbf{6}} + \dfrac{\textbf{1}}{\textbf{6}} + \dfrac{\textbf{1}}{\textbf{6}} + \dfrac{\textbf{1}}{\textbf{6}} + \dfrac{\textbf{1}}{\textbf{6}} = \dfrac{\textbf{5}}{\textbf{6}}$ \\ \newline $\therefore 3,00 + 3,00 + 3,00 + 3,00 + 3,00 = 15,00$ ou ainda $3\cdot5$ \\
			
			\item A pizza toda? Custa R\$$18,00$ ou ainda $3 \cdot 6$
		\end{enumerate}
	
	\item Se $\dfrac{\textbf{3}}{\textbf{7}}$ do que eu tenho são R\$$195,00$, a quanto corresponde $\dfrac{\textbf{4}}{\textbf{5}}$ do que eu tenho? \\ Cada parte do todo é representado por $\dfrac{\textbf{1}}{\textbf{7}} \Rightarrow \dfrac{\textbf{1}}{\textbf{7}} + \dfrac{\textbf{1}}{\textbf{7}} + \dfrac{\textbf{1}}{\textbf{7}} = \dfrac{\textbf{3}}{\textbf{7}}$ que então pode ser encontrada com $\dfrac{\textbf{195,00}}{\textbf{3}} \therefore \dfrac{\textbf{1}}{\textbf{7}} = 65$. \\
	Agora para saber quanto eu tenho: $\textbf{65,00}\cdot\textbf{7} \therefore \dfrac{\textbf{7}}{\textbf{7}} = 445$. \\
	O próximo passo é saber quanto são $\dfrac{\textbf{4}}{\textbf{5}}$ do que eu tenho (R\$$445,00$), \\
	$445,00\cdot\dfrac{\textbf{1}}{\textbf{5}} = 91,00 \therefore $ para $\dfrac{\textbf{4}}{\textbf{5}}$ multiplica-se $91,00\cdot4 = 346,00$ \\
	Então, $\dfrac{\textbf{4}}{\textbf{5}}$ corresponde a R\$$346,00$
	
\end{enumerate}

	\section{Adição e subtração de números fracionários}
	
	Recomenda-se que após a leitura desse item o estudante retorne para a \textbf{introdução a fração} a fim de melhor entender os processos de classificações das frações.
	
	Quando os \textbf{denominadores são iguais} basta somar os numeradores e preservar os denominadores.
	Dessa forma o que se tem é:
		\begin{enumerate}[label=\alph*)]
			\item $\dfrac{\textbf{4}}{\textbf{7}} + \dfrac{\textbf{2}}{\textbf{7}} = \dfrac{\textbf{6}}{\textbf{7}}$
			\item $\dfrac{\textbf{1}}{\textbf{2}} + \dfrac{\textbf{4}}{\textbf{2}} = \dfrac{\textbf{5}}{\textbf{2}}$
		\end{enumerate}
	
	\noindent Quando os \textbf{denominadores são diferentes} é preciso obter denominadores equivalentes ao \textbf{mmc} dos denominadores. Isso é feito da seguinte forma:
	\begin{enumerate}[label=\alph*)]
		\item $\dfrac{\textbf{3}}{\textbf{2}} + \dfrac{\textbf{4}}{\textbf{5}} = \dfrac{3\cdot\textbf{5} + 4\cdot\textbf{2}}{\textbf{10}} = \dfrac{\textbf{15} + \textbf{8}}{\textbf{10}} = \dfrac{\textbf{23}}{\textbf{10}}$ onde $10$ é o produto da fatoração para o \\
		
		\textbf{mínimo múltiplo comum} entre $2$ e $5$:\\		
		$\begin{array}{cc|cc}
			2, & 5 & 2 & \\ 
			1, & 5 & 5 & \\ 
			1, & 1 &  & 2 \cdot 5 = 10 \\
			\hline  
		\end{array}$ \\ \newline
		
			\item $\dfrac{\textbf{1}}{\textbf{9}} + \dfrac{\textbf{2}}{\textbf{8}} = \dfrac{1\cdot\textbf{8} + 2\cdot\textbf{9}}{\textbf{72}} = \dfrac{\textbf{8} + \textbf{18}}{\textbf{72}} = \dfrac{\textbf{26}}{\textbf{72}} = \dfrac{\textbf{26}\div\textbf{2}}{\textbf{72}\div\textbf{2}} = \dfrac{\textbf{13}}{\textbf{36}}$
			
			onde $72$ é o produto da fatoração para o \textbf{mínimo múltiplo comum} entre $9$ e $8$:			
		$\begin{array}{cc|cc}
		9, & 8 & 2 & \\ 
		9, & 4 & 2 & \\
		9, & 2 & 2 & \\
		9, & 1 & 3 & \\
		3, & 1 & 3 & \\
		1, & 1 &  & 2^{3} \cdot 3^{2} = 8 \cdot 9 = 72 \\
		\hline
		\end{array}$
	\end{enumerate}
	
	\subsection{Exemplo}
	\begin{enumerate}
	\item Encontre os resultados dos cálculos abaixo:
		\begin{enumerate}[label=\alph*)]
			\item  $\dfrac{\textbf{7}}{\textbf{5}} + \dfrac{\textbf{3}}{\textbf{5}} = \dfrac{\textbf{7} + \textbf{3}}{\textbf{5}} = \dfrac{\textbf{10}}{\textbf{5}} = \textbf{2}$\\
			
			\item  $\dfrac{\textbf{4}}{\textbf{8}} + \dfrac{\textbf{2}}{\textbf{8}} = \dfrac{\textbf{4} + \textbf{2}}{\textbf{8}} = \dfrac{\textbf{6}}{\textbf{8}} = \dfrac{\textbf{6}\div\textbf{2}}{\textbf{8}\div\textbf{2}} = \dfrac{\textbf{3}}{\textbf{4}}$\\
			
			\item  $\dfrac{\textbf{3}}{\textbf{4}} + \dfrac{\textbf{6}}{\textbf{12}} = \dfrac{3\cdot\textbf{3} + 6\cdot\textbf{1}}{\textbf{12}} = \dfrac{\textbf{9} + \textbf{6}}{\textbf{12}} = \dfrac{\textbf{15}}{\textbf{12}} = \dfrac{\textbf{15}\div\textbf{3}}{\textbf{12}\div\textbf{3}} = \dfrac{\textbf{5}}{\textbf{4}}$ \\
			
			onde $12$ é o produto da fatoração para o \textbf{mínimo múltiplo comum} entre $4$ e $12$:
			% decomposição em fatores primos
			
			$\begin{array}{cc|cc}
			4, & 12 & 2 & \\ 
			2, & 6 & 2 & \\
			1, & 3 & 3 & \\
			1, & 1 &  & 2^{2} \cdot 3 = 4 \cdot 3 = 12 \\
			\hline 
			\end{array}$ \\
				\end{enumerate}
						
			Outra forma de fazer a letra \textbf{c} é seguinte; \\ 
			para: \\
			
			 $ \dfrac{\textbf{3}}{\textbf{4}} + \dfrac{\textbf{6}}{\textbf{12}} = \dfrac{\textbf{12}\cdot\textbf{3} + \textbf{6}\cdot\textbf{4}}{\textbf{4}\cdot\textbf{12}} = \dfrac{\textbf{36} + \textbf{24}}{\textbf{48}} = \dfrac{\textbf{60}}{\textbf{48}} = \dfrac{\textbf{60}\div\textbf{12}}{\textbf{48}\div\textbf{12}} = \dfrac{\textbf{5}}{\textbf{4}}$ \\
			
		
	\end{enumerate}

\section{Multiplicação e divisão de números fracionários}

Na multiplicação de fração devemos multiplicar numerador com numerador e denominador com denominador, assim:
\begin{enumerate}[label=\alph*)]
	\item $\dfrac{\textbf{8}}{\textbf{3}} \cdot \dfrac{\textbf{2}}{\textbf{3}} = \dfrac{\textbf{16}}{\textbf{9}}$
	\item $\dfrac{\textbf{-5}}{\textbf{2}} \cdot \dfrac{\textbf{4}}{\textbf{3}} = \dfrac{\textbf{-20}}{\textbf{6}} = - \dfrac{\textbf{10}}{\textbf{3}}$ \\ 
\end{enumerate}

\noindent Na divisão de números fracionários devemos multiplicar a $1ª$ fração com o inverso da $2ª$, assim:
\begin{enumerate}[label=\alph*)]
	\item $\dfrac{\dfrac{\textbf{8}}{\textbf{3}}}{\dfrac{\textbf{3}}{\textbf{4}}} = \dfrac{\textbf{8}}{\textbf{3}} \cdot \dfrac{\textbf{4}}{\textbf{3}} = \dfrac{\textbf{32}}{\textbf{9}}$
	
	\item $\dfrac{\textbf{2}}{\textbf{3}} \div \dfrac{\textbf{1}}{\textbf{4}} = \dfrac{\textbf{2}}{\textbf{3}} \cdot \textbf{4} = \dfrac{\textbf{8}}{\textbf{3}}$
	
	\item $\dfrac{\textbf{5}}{\textbf{2}} \div \textbf{4} = \dfrac{\textbf{5}}{\textbf{2}} \cdot \dfrac{\textbf{1}}{\textbf{4}} = \dfrac{\textbf{5}}{\textbf{8}}$
	
	\item $\dfrac{\textbf{6}}{\frac{\textbf{3}}{\textbf{4}}} = \textbf{6} \cdot \dfrac{\textbf{4}}{\textbf{3}} = \dfrac{\textbf{24}}{\textbf{3}} = \textbf{8}$
\end{enumerate}

\section{Potenciação e radiciação de números fracionários}

Na potenciação quando elevamos um número fracionário a um determinado expoente, estamos elevando o numerador e o denominador a esse expoente, assim:
\begin{enumerate}[label=\alph*)]
	\item $\left(\dfrac{\textbf{4}}{\textbf{3}}\right)^{2} = \dfrac{\textbf{4}^{2}}{\textbf{3}^{2}} = \dfrac{\textbf{16}}{\textbf{9}}$
	\item $\left(\dfrac{\textbf{2}}{\textbf{3}}\right)^{3} = \dfrac{\textbf{2}^{3}}{\textbf{3}^{3}} = \dfrac{\textbf{8}}{\textbf{27}}$ \\ 
\end{enumerate}

\noindent Na radiciação quando buscamos a raiz de número fracionário estamos extraindo a raiz do numerador e do denominador, assim:
\begin{enumerate}[label=\alph*)]
	\item $\sqrt{\dfrac{\textbf{25}}{\textbf{81}}} = \dfrac{\sqrt{\textbf{25}}}{\sqrt{\textbf{81}}} = \dfrac{\sqrt[\cancel{}]{\textbf{5}\cancel{^2}}}{\sqrt[\cancel{}]{\textbf{9}\cancel{^2}}} = \dfrac{\textbf{5}}{\textbf{9}}$
	\item $\sqrt{\textbf{1,44}} = \sqrt{\dfrac{\textbf{144}}{\textbf{100}}} = \dfrac{\sqrt[\cancel{}]{\textbf{12}\cancel{^2}}}{\sqrt[\cancel{}]{\textbf{10}\cancel{^2}}} = \dfrac{\textbf{12}}{\textbf{10}} = \dfrac{\textbf{6}}{\textbf{5}}$ \\
\end{enumerate}

\section{Critérios de divisibilidade}

\begin{enumerate}
	\item \textbf{Divisibilidade por 2} \\
	Um número \textbf{$\aleph$} é divisível po $2$ quando ele é par

\item \textbf{Divisibilidade por 3} \\
Um número é divisível por $3$ quando a soma dos valores absolutos de seus algorismos for divisível por $3$

\textbf{Exemplo}
	\begin{itemize}
		\item $234$ é divisível por $3$ pois a soma de de $2+3+4 = 9$ e como $9$ é divisível por $3$, então $234$ também. 
	\end{itemize} 

\item \textbf{Divisibilidade por 4} \\
Um número é divisível por $4$ quando termina em $00$ ou o número formado pelos seus dois últimos algorismos da direita for divisível por $4$

	\textbf{Exemplo}
	\begin{itemize}
		\item $1600$ é divisível por $4$ pois termina com $00$;
		\item $4116$ é divisível por $4$ pois termina com $16$ que é divisível por $4$;
		\item $1324$ é divisível por $4$ pois termina com $24$ que é divisível por $4$;
		\item $3850$ não é divisível por $4$ pois não termina com $00$ e $50$ não é divisível por $4$;
	\end{itemize}

\item \textbf{Divisibilidade por 5} \\
Um número é divisível por $5$ quando termina em $0$ ou $5$

\textbf{Exemplo}
\begin{itemize}
	\item $55$ é divisível por $5$ pois termina com $5$;
	\item $610$ é divisível por $5$ pois termina com $0$;
\end{itemize}

\item \textbf{Divisibilidade por 6} \\
Um número é divisível por $6$ quando é divisível por $2$ e por $3$

\textbf{Exemplo}
\begin{itemize}
	\item $6$ é divisível por $2$ e por $3$;
	\item $36$ é divisível por $2$ e por $3$;
	\item $5214$ é divisível por $2$ e por $3$;
\end{itemize}

\end{enumerate}

\section{Números decimais}

As frações decimais possuem os \textbf{denominadores como potências de $ \textbf{10} $}.\\

\begin{multicols}{3}
	$ \dfrac{3}{10} $ \\ o denominador é $ 10^1 $\\
	
	\columnbreak
	$ \dfrac{17}{100} $ \\ o denominador é $ 10^2 $\\
	
	\columnbreak
	$ \dfrac{54}{1000} $ \\ o denominador é $ 10^3 $\\
	
\end{multicols}
 O \textbf{número decimal} indica um número que não é inteiro \\
 
 \begin{multicols}{3}
 	$ 5,28 $
 	
 	\columnbreak 
 	$ 0,374 $
 	
 	\columnbreak 
 	$ 54,08 $ 	
 \end{multicols}

Para transformar um numeral decimal em fração decimal desloca-se a virgula para a \textbf{direita} e escreve o \textbf{denominador} como potencia de $ 10^n $, onde $ n $ é a quantidade de casal decimais que a vírgula foi deslocada.

\begin{itemize}
	\item $ 47,32 \longrightarrow \dfrac{4732}{10^2} $ onde $ 10^2 = 100 $ e, portanto, $ \dfrac{4732}{100} $
	\item $ 0,023 \longrightarrow \dfrac{23}{10^3} $ onde $ 10^3 = 1000 $ e, portanto, $ \dfrac{23}{1000} $ \\
\end{itemize}

Para transformar um fração decimal em numeral decimal desloca-se a \textbf{virgula} para a \textbf{esquerda} a quantidade de casas decimais indicada pelo denominador.

\begin{itemize}
	\item $ \dfrac{45}{1000} \longrightarrow 0,045 $ onde a vírgula foi deslocada para a esquerda $ 3 $ casas decimal
	\item $ \dfrac{3548}{100} \longrightarrow 35,48 $ onde a vírgula foi deslocada para a esquerda $ 2 $ casas decimal
\end{itemize}

\subsection{Propriedades dos números decimais}
	\begin{enumerate}[label=\alph*)]
		\item A quantidade de zeros após a vírgula que ocupa a posição mais à direita é desprezada.\\
		$ 4,23 $ ou $ 4,230 $ ou $ 4,2300 $ é o mesmo que $ 4,32 $
		
		\item Na multiplicação de um número fracionário a vírgula será deslocada para a direita na mesma quantidade do expoente da dezena
		\begin{multicols}{3}
			$ 4,236 \cdot 10 = 42,36 $
			
			\columnbreak
			$ 4,236 \cdot 100 = 423,6 $
			
			\columnbreak
			$ 0,048 \cdot 1000 = 48,0 $ ou conforme a regra do item \textbf{a)} $ 48 $
			
		\end{multicols}
		
		\item Na divisão de um número fracionário a vírgula será deslocada para a esquerda na mesma quantidade do expoente da dezena
		\begin{multicols}{3}
			$ 361,7 \div 10 = 36,17 $
			
			\columnbreak
			$ 361,7 \div 100 = 3,617 $
			
			\columnbreak
			$ 35,6 \div 1000 = 0,0356 $
			
		\end{multicols}
	\end{enumerate}

\section{Fração geratriz}
Transformar uma dízima periódica em fração geratriz. A dízima periódica pode ser simples ou composta.
\begin{itemize}
	\item Simples: $ 1,\color{red}42424242 \hdots $ ou na notação $ 1,\bar{42} $
	\item Composta: $ 1,\color{blue}3\color{red}42424242 \hdots $ ou na notação $ 1,3\bar{42} $
\end{itemize}

Na representação de um dízima simples $ (D) $, deve-se ter em mente o que cada número significa em relação a vírgula. Observe esse modelo: $ D = (a,\color{red}bbb \hdots) $. Vê-se que  $ a $ representa o número inteiro e $ \color{red}b $ a parte decimal. Também pode ser escrito assim: \\
$ D = (a + 0,\color{red}bbb \hdots) $.

Na representação de um dízima composta $ (Dc) $, também tendo a vírgula como elemento de referência, o modelo fica assim: $ Dc = (a,\color{blue}i\color{red}bbb \hdots) $. O que muda em relação a dízima simples é que $ \color{blue}i $ não tende ao infinito repetidamente e antecede o $ \color{red}b $ da parte decimal. Também pode ser escrito assim: $ Dc = (a + 0,\color{blue}i\color{red}bbb \hdots) $. \\

\newpage

Vou chamar o $ a $ de \textit{parte inteira}; o $ \color{blue}i $ de \textit{intruso} e o $ \color{red}bbb $ de \textit{período}. \\

Exemplo com $ 10^1 $:
\begin{multicols}{2}
		\setlength{\columnseprule}{.5pt}
	$ 0,222222 \hdots  \longrightarrow 0,\bar{2} $; \\
	chamando isso de $ x $, tem-se: $ x = 0,\bar{2} $; \\
	multiplica-se os dois lados por $ 10^1 $, temos: \\
	$ 10x = 2,\bar{2} \therefore $ \\
	$ 10x = 2,\bar{2} $, pode ser escrito assim: \\ $ 10x = 2 + 0,\bar{2} $\\
	subtrai os dois lados por $ x $, obtém-se: \\ $ 10x - x = 2 + x - x $, daí fica: \\	
	$ 9x = 2 \longrightarrow x = \dfrac{2}{9} $
	
	\columnbreak
	Agora é possível entender que após identificar o expoente da dezena, para cada casa decimal que neste exemplo é apenas o número $ 2 $, basta subtrair $ 1 $ do denominador e subtrair o inteiro do numerador após ter a vírgula deslocada uma casa para a direita, ou seja $ a - 10^n({\color{blue}D}) $:\\
	
	$ 0,\bar{2} \longrightarrow \dfrac{2 - 0}{10-1} $\\
	 A fração geratriz de $ 0,\bar{2} $ é $ \dfrac{2}{9} $
	
\end{multicols}

\textbf{Atenção}: Sempre que uma dízima tiver seu período com $ 9 $ tendendo a infinito o resultado será finito.

$ 0,999 \hdots \longrightarrow 0,\bar{9} $; \\
	chamando isso de $ x $, tem-se: $ x = 0,\bar{9} $; \\
	multiplica-se os dois lados por $ 10^1 $, temos: \\
	$ 10x = 9,\bar{9} \therefore $ \\
	$ 10x = 9,\bar{9} $, pode ser escrito assim: \\ $ 10x = 9 + 0,\bar{9} $\\
	subtrai os dois lados por $ x $, obtém-se: \\ $ 10x - x = 9 + x - x $, daí fica: \\
	$ 9x = 9 \longrightarrow x = \dfrac{9}{9} = 1 $ \\
	
	Com a visão obtida pela entendimento da operação pode-se fazer de maneira direta:
	
	$ 0,\bar{9} \longrightarrow \dfrac{9 - 0}{10-1} $, portanto a fração geratriz de $ 0,\bar{9} $ é $ \dfrac{9}{9} = 1 $ \\
	
	Veja o exemplo com um número inteiro maior do que \textit{zero}.\\
	
$ 2,444 \hdots \longrightarrow 2,\bar{4} $; \\
	chamando isso de $ x $, tem-se: $ x = 2,\bar{4} $; \\
	multiplica-se os dois lados por $ 10^1 $, temos: \\
	$ 10x = 24,\bar{4} \therefore $ \\
	$ 10x = 24,\bar{4} $, pode ser escrito assim: \\ $ 10x = 24 + 0,\bar{4} $\\
	subtrai os dois lados por $ x $ (lembre-se de que $ x = 2,\bar{4} $), obtém-se: \\ $ 10x - x = 24 + 0,\bar{4} - 2,\bar{4} $, daí monte a subtração pra enxergar melhor: \\
	$
		\begin{array}{r l}
		\multicolumn{2}{r}{10x = 24,\bar{4}} \\ 
		\multicolumn{1}{r}{-x = \;\;\: 2,\bar{4}} \\ \cline{1-2}
		\multicolumn{2}{r}{9x = 22 \;\;\:\;\;} \\
		\end{array}
	$
		
	$ 9x = 22 \longrightarrow x = \dfrac{22}{9} $ \\
	
	Pela maneira direta fica assim:
	
	$ 2,\bar{4} \longrightarrow \dfrac{24 - 2}{10-1} $, portanto a fração geratriz de $ 2,\bar{4} $ é $ \dfrac{22}{9} $ \\

Outro exemplo, agora com $ 10^2 $:
\begin{multicols}{2}
	\setlength{\columnseprule}{.5pt}
	$ 0,484848 \hdots  \longrightarrow 0,\bar{48} $;
	chamando isso de $ x $, tem-se: $ x = 0,\bar{48} $;
	deslocar a vírgula até o período: $ 48,\bar{48} \therefore $ \\
	$ 100x = 48,\bar{48} \longrightarrow 100x = 48 + 0,\bar{48} $\\
	$ 100x = 48 + x $ \\
	$ 100x - x = 48 + x - x $ \\
	$ 99x = 48 $ \\
	$ x = \frac{48}{99} $
	
	\columnbreak
	Pela forma direita, têm-se:\\
	
	$ 0,\bar{48} \longrightarrow \dfrac{48 - 0}{100-1} $\\
	A fração geratriz de $ 0,\bar{48} $ é $ \dfrac{48}{99} $
	
\end{multicols}

Veja o exemplo com um número inteiro maior do que \textit{zero} e período com dezena.\\
	
$ 1,141414 \hdots \longrightarrow 1,\bar{14} $; \\
	chamando isso de $ x $, tem-se: $ x = 1,\bar{14} $; \\
	multiplica-se os dois lados por $ 10^1 $, temos: \\
	$ 10x = 11,\bar{41} \therefore $ \\
	$ 10x = 11,\bar{41} $, pode ser escrito assim: \\ $ 10x = 11 + 0,\bar{41} $\\
	ao tentar subtrai os dois lados por $ x $ (lembre-se de que $ x = 1,\bar{14} $), obtém-se: \\ $ 10x - x = 11 + 0,\bar{41} - 1,\bar{14} $, e dá para perceber que a subtração dos períodos é diferente de \textit{zero}. Para resolver isso é importante tornar a multiplicar por $ 10^1 $, quantas vezes forem necessárias até que o período fique igual ao período original. Então ficará assim para $ 10^2 $: \\
	$ 100x - x = 114 + 0,\bar{14} - 1,\bar{14} $ daí monte a subtração pra enxergar melhor: \\
	$
		\begin{array}{r l}
		\multicolumn{2}{r}{100x = 114,\bar{14}} \\ 
		\multicolumn{1}{r}{\;\:-x = \;\;\:\;\; 1,\bar{14}} \\ \cline{1-2}
		\multicolumn{2}{r}{99x = 113 \;\;\:\;\;\;\:} \\
		\end{array}
	$
		
	$ 99x = 113 \longrightarrow x = \dfrac{113}{99} $ \\
	
	Pela maneira direta fica assim:
	
	$ 1,\bar{14} \longrightarrow \dfrac{114 - 1}{100-1} $, portanto a fração geratriz de $ 1,\bar{14} $ é $ \dfrac{113}{99} $ \\
	
	Também é possível resolver com fração imprópria pois sabemos que o denominador sempre será $ 10^n -1 $. Acrescenta-se a está regra o fato de que $ n $ deverá ter o cumprimento suficiente para alcançar o próximo inteiro do período $ \color{red}b $.
	
	\begin{itemize}
		\item se for dízima periódica de $ 10^1 $ será $ 10 - 1 $
		\item se for dízima periódica de $ 10^2 $ será $ 100 - 1 $
		\item se for dízima periódica de $ 10^3 $ será $ 1000 - 1 $ ...
	\end{itemize}
	
	Então $ D = (a + 0,\color{red}bbb \hdots) $ pode ser resolvida com $ D = a + \dfrac{\color{red}b}{\color{blue}10^n - 1} \Longleftrightarrow {\color{blue}10^n - 1} \cdot a + \color{red}b $
	
	\begin{itemize}
		\item $ 0,\bar{4} $ para $ 10^1 \Rightarrow 0 + \dfrac{\color{red}4}{\color{blue}10^1 - 1} \longrightarrow \dfrac{4}{9} $;
		\item $ 0,\bar{9} $ para $ 10^1 \Rightarrow 0 + \dfrac{\color{red}9}{\color{blue}10^1 - 1} \longrightarrow \dfrac{9}{9} = 1 $;
		
		\item $ 0,\bar{2} $ para $ 10^2 \Rightarrow 0 + \dfrac{\color{red}22}{\color{blue}100^1 - 1} \longrightarrow \dfrac{22}{99} $;
		\item $ 0,\bar{9} $ para $ 10^2 \Rightarrow 0 + \dfrac{\color{red}99}{\color{blue}100^1 - 1} \longrightarrow \dfrac{99}{99} = 1 $;
		
		\item $ 1,\bar{2} $ para $ 10^1 \Rightarrow 1 + \dfrac{\color{red}2}{\color{blue}10^1 - 1} \longrightarrow \dfrac{11}{9} $;
		\item $ 2,\bar{1} $ para $ 10^1 \Rightarrow 2 + \dfrac{\color{red}1}{\color{blue}10^1 - 1} \longrightarrow \dfrac{19}{9} $;
		
		\item $ 1,\bar{12} $ para $ 10^2 \Rightarrow 1 + \dfrac{\color{red}12}{\color{blue}100^1 - 1} \longrightarrow \dfrac{111}{99} $ ($n$ deve ser um múltiplo de $2$);
		\item $ 1,\bar{9} $ para $ 10^2 \Rightarrow 1 + \dfrac{\color{red}99}{\color{blue}100^1 - 1} \longrightarrow \dfrac{198}{99} = 2 $;
		
		\item $ 1,\bar{001} $ para $ 10^3 \Rightarrow 1 + \dfrac{\color{red}1}{\color{blue}1000^1 - 1} \longrightarrow \dfrac{1000}{999} $ ($n$ deve ser um múltiplo de $3$);
		\item $ 0,\bar{1} $ para $ 10^3 \Rightarrow 0 + \dfrac{\color{red}111}{\color{blue}1000^1 - 1} \longrightarrow \dfrac{111}{999} $;
		\item $ 3,\bar{9} $ para $ 10^3 \Rightarrow 3 + \dfrac{\color{red}999}{\color{blue}1000^1 - 1} \longrightarrow \dfrac{2997}{999} = 3 $;
		
		\item $ 1,\bar{13} $ para $ 10^4 \Rightarrow 1 + \dfrac{\color{red}1313}{\color{blue}10000^1 - 1} \longrightarrow \dfrac{11312}{9999} $ ($n$ deve ser um múltiplo de $2$);
		\item $ 1,\bar{05} $ para $ 10^4 \Rightarrow 1 + \dfrac{\color{red}0505}{\color{blue}10000^1 - 1} \longrightarrow \dfrac{10504}{9999} $ ($n$ deve ser um múltiplo de $2$);
	\end{itemize}


Outro exemplo, agora com $ 10^1 $ em uma dízima periódica composta:
\begin{multicols}{2}
	\setlength{\columnseprule}{.5pt}
	$ 0,1787878 \hdots  \longrightarrow 0,1\bar{78} $;
	chamando isso de $ x $, tem-se: $ x = 0,1\bar{78} $;
	deslocar a vírgula até o período: $ 1,\bar{78} $, daí $ x $ passa a ser $ 0,\bar{78} $. Deve-se resolver o $ x $ conformes a demonstrado para dízima periódica simples e depois o seguinte: \\
	$ 10x = 1,\bar{78} \longrightarrow 10x = 1 + 0,\bar{78} $\\
	$ 10x = 1 + 0,78 $ (0,78 gera a fração geratriz $ \frac{78}{99} $), então \\
	$ 10x = 1 + \frac{78}{99} $\\
	$ 10x = \frac{177}{99} $\\
	$ x = \frac{\frac{177}{99}}{10} \longrightarrow x = \frac{177}{99}\cdot\frac{1}{10} $\\
	\noindent$ x = \frac{177}{990} $
	
	\columnbreak
	Agora é possível entender que, além das propriedades já identificadas, deve-se, para o numerador: preservar todos os números até o período; subtrair dele o nº que está entre a vírgula e a dízima;\\
	para o denominador: identificar o expoente da dezena, para cada casa decimal, acrescentar a direita um zero para cada número que antecede o período\\
	
	$ 0,1\bar{78} \longrightarrow \dfrac{178 - 1}{(100-1) \cdot 10} $\\
	A fração geratriz de $ 0,1\bar{78} $ é $ \dfrac{177}{990} $
	
\end{multicols}

Outro exemplo, agora com $ 10^1 $ em uma dízima periódica simples:
	
	$ 2,\bar{4} \longrightarrow 2 + \dfrac{4}{9} \longrightarrow 9 \cdot 2 + 4 = \dfrac{22}{9} $\\
	A fração geratriz de $ 2,\bar{4} $ é $ \dfrac{22}{9} $\\

Outro exemplo, agora com $ 10^2 $ em uma dízima periódica composta:

$ 0,12\bar{3} \longrightarrow \frac{123-12}{900} \longrightarrow \frac{111}{900} $\\
A fração geratriz de $ 0,12\bar{3} $ é $ \frac{111}{900} $\\

Outro exemplo, agora com $ 10^1 $ em uma dízima periódica composta:

$ 0,4\bar{82} \longrightarrow \frac{482-4}{990} \longrightarrow \frac{478}{990} = \frac{239}{495} $\\
A fração geratriz de $ 0,4\bar{82} $ é $ \frac{239}{495} $

\section{Números primos}
Um número é primo quando tem apenas $2$ divisores diferentes: o $1$ e ele mesmo.\\
Para saber se um número é primo divide-se esse número pelos números primos até que o \textbf{quociente seja menor do que o divisor} e o \textbf{resto diferente de zero}.

\begin{itemize}
	\item $1$ não é número primo $\Rightarrow$ divisores: $1$;
	\item $2$ é o único número primo par $\Rightarrow$ divisores: $1$ e $2$;
	\item $3$ é número primo $\Rightarrow$ divisores: $1$ e $3$;
	\item $113$ é número primo $\Rightarrow$ divisores: $1$ e $113$;
\end{itemize}

O número $113$ é um número primo pois na divisão pelos números primos $2, 3, 5, 7, 11, \dots$ quando dividido por $11$ o que se vê é que o \textbf{quociente é menor do que o divisor} e o \textbf{resto é diferente de zero}:\\

% conta de dividir

	$
	\begin{array}{l l r r}
	\multicolumn{2}{r}{113} \vline & \multicolumn{2}{c}{11} \\ \cline{3-4}
	\multicolumn{2}{l}{-110} & \multicolumn{2}{l}{10} \\ \cline{1-2}
	\multicolumn{2}{r}{003} &  \\
	\end{array}
	$
	

